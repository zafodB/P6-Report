\subsection{Fiat  currency}
% 
Currency, as the medium of exchange for goods and services, is the basis of trade. Since the history of humanity, we have always engaged in some sort of a trade. Whenever we needed a commodity we did not have, we needed to get this commodity by trading it for something else. A farmer could for example visit his neighbour locksmith and exchange fresh eggs for a pair of horseshoes. This exchange of commodities was referred to as \textit{barter trade}~\cite{Sullivan2009BarterEconomics}. Barter trade, however was not always suitable, since it required the alignment of wants and was not very scale-able for large transactions~\cite{Carroll2015CreatingExchange}. To overcome these limitations, money was invented. Money is a more standardised means of exchange, that is generally accepted. Any scarce commodity could become money, if it became widely accepted as a form of payment. For example, in some countries salt was used as money to make trades. This is not the same as barter trade, in which any commodity at hand can be used. But money is not necessarily the same as `currency'. Currency is a more specific form of money, which is issued and controlled by the government and has the form of notes and coins\footnotemark. Euros and US dollars are examples of currency.
% 
\footnotetext{\url{https://www.investopedia.com/terms/c/currency.asp}, accessed 22-03-2018}

Until 1971, every US dollar could be exchanged for its respective value in gold. This system is known as a ``representative currency'', since the entire currency is represented by the amount of gold the government posses. This is also called the ``Gold standard'' and is currently rarely used. In opposition to the representative currency stands the fiat currency. 

A fiat currency is a currency that is issued and controlled by the government and is declared a legal tender in respective country. However, unlike the representative currency, it is not backed by any physical commodity and its value is simply derived from the supply-demand relationship\footnotemark. Most of today's currencies, including US Dollars, Euros, Yens or Danish Crowns are fiat currencies.
% 
\footnotetext{\url{https://www.investopedia.com/terms/f/fiatmoney.asp}, accessed 22-03-2018}