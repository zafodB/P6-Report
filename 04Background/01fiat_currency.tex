\subsection{Fiat  currency}
% 
Currency, as the medium of exchange for goods and services, is the basis of trade. Since the history of humanity, we have always engaged in some sort of a trade. Whenever we needed a commodity we do not posses, we needed to get this commodity by trading it for something else. This exchange of commodities was referred to as barter trade. Barter trade, however was not always suitable, since it required the alignment of wants and was not very scalable for large transactions~\cite{Carroll2015CreatingExchange}. To overcome these limitations, money was invented. Money can be seen as opposition to barter trade. In the beginning, cattle, salt or precious metals were used as 'comodity money', later coins were invented. But money is not necessarily the same as `currency'.  Currency is more specific form of money. Currency, in traditional meaning is issued and controlled by the government\footnotemark.
% 
\footnotetext{\url{https://www.investopedia.com/terms/c/currency.asp}, accessed 22-03-2018}

Until 1971, every US dollar could be exchanged for its respective value in gold. This system is known as a ``representative currency'', since the entire currency is represented by the amount of gold the government posses. This is also called ``Gold standard'' and is currently rarely used. In opposition to representative currency stands the fiat currency. 

Fiat currency is a currency that is issued and controlled by the government and is declared a legal tender in respective country. However, unlike the representative currency, it is not backed by any physical commodity and its value is simply derived from the supply-demand relationship\footnotemark. Most of today's currencies, including US Dollars, Euros, Yens or Danish Crowns are fiat currencies.
% 
\footnotetext{\url{https://www.investopedia.com/terms/f/fiatmoney.asp}, accessed 22-03-2018}