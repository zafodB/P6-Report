\section{Background}\label{sec:background}
% 
In this section, we explain some of the main concepts used in this project. This section is needed to establish a common understanding behind these topics, so that unambiguous discussion can follow.

First, we briefly talk about the concept of currency and its exchange. Afterwards we define what a cryptocurrency is and how it differs from a fiat currency. At the end of this chapter, we describe two most popular cryptocurrencies Bitcoin and Ethereum.
% 
\subsection{Fiat  currency}
% 
Currency, as the medium of exchange for goods and services, is the basis of trade. Since the history of humanity, we have always engaged in some sort of a trade. Whenever we needed a commodity we did not have, we needed to get this commodity by trading it for something else. A farmer could for example visit his neighbour locksmith and exchange fresh eggs for a pair of horseshoes. This exchange of commodities was referred to as \textit{barter trade}~\cite{Sullivan2009BarterEconomics}. Barter trade, however was not always suitable, since it required the alignment of wants and was not very scaleable for large transactions~\cite{Carroll2015CreatingExchange}. To overcome these limitations, money was invented. Money is a more standardised means of exchange, that is generally accepted. Any scarce commodity could become money, if it becomes widely accepted as a form of payment. For example, in some countries salt was used as money to make trades. This is not the same as barter trade, in which any commodity at hand can be used. But money is not necessarily the same as `currency'. Currency is a more specific form of money, which is issued and controlled by the government and has the form of notes and coins\footnotemark. Euros and US dollars are examples of currency.
% 
\footnotetext{\url{https://www.investopedia.com/terms/c/currency.asp}, accessed 22-03-2018}

Until 1971, every US dollar could be exchanged for its respective value in gold. This system is known as a \textit{representative currency}, since the entire currency is represented by the amount of gold the government posses. This is also called the \textit{Gold standard} and is currently rarely used. In opposition to the representative currency stands the fiat currency. 

A fiat currency is a currency that is issued and controlled by the government and is declared a legal tender in respective country. However, unlike the representative currency, it is not backed by any physical commodity and its value is simply derived from the supply-demand relationship\footnotemark. Most of today's currencies, including US Dollars, Euros, Yens or Danish Crowns are fiat currencies.
% 
\footnotetext{\url{https://www.investopedia.com/terms/f/fiatmoney.asp}, accessed 22-03-2018}
% 
\subsection{Cryptocurrency}
% 
Merriam-Webster dictionary defines cryptocurrency as a \textit{``form of currency that only exists digitally, that usually has no central issuing or regulating authority, but instead uses a decentralised system to record transactions and manage the issuance of new units, and that relies on cryptography to prevent counterfeiting and fraudulent transactions''}\footnotemark.
% 
\footnotetext{\url{https://www.merriam-webster.com/dictionary/cryptocurrency}, accessed 27-03-2018}

Digital currency means that we can only access it in a digital world. This is opposed to fiat currencies, which have both physical representation of the currency (coins and notes) and digital representation (money in a bank). Technology of cryptography assures that the currency will only be used in the intended way. Fraud cases, such as producing additional fake currency, double spending, or unauthorised transfer need to be prevented. With the physical cash, measures can be taken to prevent tampering with the notes and coins, such as watermarks, special print or holograms. In the digital world, these measures need to be enforceable and verifiable by the computer.

There are numerous technical characteristics of cryptocurrencies. However, the most prominent ones, as listed by many research works~\cite{Lansky2018PossibleCryptocurrencies} are the following:
\begin{enumerate}[noitemsep]
    \item The system does not require a central authority, distributed entities achieve consensus on its state.
    \item The system keeps an overview of cryptocurrency units and their ownership.
    \item The system defines whether new cryptocurrency units can be created. If new cryptocurrency units can  be  created,  the  system  defines  the  circumstances  of  their  origin  and  how  to  determine  the ownership of these new units.
    \item Ownership of cryptocurrency units can be proved exclusively cryptographically.
    \item The system allows transactions to be performed in  which ownership of the cryptographic units is changed. A transaction statement can only be issued by an entity proving the current ownership of these units.
    \item If  two  different  instructions  for  changing  the  ownership  of the  same  cryptographic  units  are simultaneously entered, the system performs at most one of them.
\end{enumerate}

Public-key encryption and hashing are the two most used cryptographic concepts used in cryptocurrencies. In several cryptocurrencies, hashing is used to maintain a consensus over the network's state and public-key cryptography is used to prove the ownership of the cryptocurrency\footnotemark.
% 
\footnotetext{Not all currencies use the public-key cryptography. For example, Fawkescoin only uses hashing for both state consensus and currency ownership~\cite{Bonneau2014FawkescoinCryptography}}

\subsubsection{Motivation for cryptocurrencies}
But cryptocurrency was not entering a blue ocean upon its inception. Today, all countries have a generally accepted currency, usually issued by central bank. This system, however, has certain limitations. The main drawback is virtually unlimited supply of the currency. Government can issue more money, as it deems necessary. The value of a fiat currency can therefore decrease drastically (e.g. the value of Zimbabwean dollar dropped by hundreds of percents in 2006 \footnotemark). Secondary, the government could use its ability to control the currency to influence anti-governmental organisations or groups by limiting their access to their funds. This can be a problem for whistle-blower organisations (such as WikiLeaks) or for freedom movements in repressive regimes.
% 
\footnotetext{\url{https://www.nytimes.com/2006/05/02/world/africa/02zimbabwe.html}, accessed 28-03-18}

Cryptocurrency overcomes these problems, since there is no central authority, that could control the currency. Every user of the currency can participate in the consensus negotiation process. The issue of units of the currency is controlled by the system in a well-defined and expectable manner. The trust in the central bank is thus replaced by trust in the cryptocurrency. In the next section we will explore, how blockchain enables this decentralisation.
% 
\subsection{Blockchain}
% 
Since the inception of a digital currency Bitcoin (described in the following section) in 2009, blockchain has been part of the discussions about this novel approach to currency. Over the time, the perception shifted from seeing blockchain just as a part of cryptocurrency into seeing it as an separate, innovative, even disruptive technology. Some media mark \textit{blockchain} to be the word of the year 2017\footnotemark, while others comparable it to inception of the Web in 1990s~\cite[p. 14]{Swan2015BlockchainEconomy}.
% 
\footnotetext{\url{https://www.theguardian.com/technology/2018/jan/30/blockchain-buzzword-hype-open-source-ledger-bitcoin}, accessed 28-03-2018}

From the perspective of cryptocurrency, blockchain is a public ledger, that contains all the transactions of the cryptocurrency to date~\cite{Swan2015BlockchainEconomy}. It is distributed among all the computers participating in the consensus process. Since it contains all the past transactions (since the inception of the currency), data is never deleted from the blockchain. All changes happen only as amendments to the latest version of the blockchain.

A \textit{change} could be virtually any data. For example, it could be details of latest transactions (as is the case with Bitcoin) or newly deposited agreements (as in Quorum - enterprise level ledger \footnotemark ). This \textit{change} is called a \textit{block}. Blocks are the fundamental parts, that make up the blockchain. When a new blockchain is created, a first set of changes becomes the first block.
% 
\footnotetext{Quorum is a blockchain-based private storage for agreements. Its intended users are enterprises in the finance industry, who are trading financial derivatives and who need to reach an agreement, while maintaining acceptable level of privacy.
\begin{flushleft}
\url{http://fortune.com/2016/10/04/jp-morgan-chase-blockchain-ethereum-quorum/}, accessed 28-03-18
\url{https://www.jpmorgan.com/global/Quorum}, accessed 28-03-18
\end{flushleft}
}
% 
When more changes are made (for example, more transactions are processed), a new block is created. The creation of a new block involves computing a hash value of the previous block. This hash value is then included in the new block, together with the data comprising the change. By including the hash value of the old block in the new block, these two blocks are now linked. All the blocks in the blockchain are linked together in such fashion. Figure~\ref{fig:blockch-basics} illustrates this principle. 
% 
\begin{figure}[h]
    \centering
    \includegraphics[width=.95\textwidth]{blockchain-basics}
    \caption{The basic architecture of a blockchain. If Block \#1 is the first block in the chain, it is also referred to as the \textit{genesis block}.}
    \label{fig:blockch-basics}
\end{figure}

It is not possible to alter the past blocks in the blockchain. In order to accomplish this, we would need to find such a combination of data, that would produce the same hash. This violates the pre-image resistance property of the hash function. Alternative technique could be to change the data in block \textit{n}, then calculate new hash and include it in block \textit{n+1} and so on, recalculating every subsequent block~\cite[3]{NakamotoBitcoin:System}.

The decentralisation of the blockchain prevents this. Every participating node maintains and updates its own copy of the blockchain. When there is a dispute about the correct version of the blockchain, the version that is present on most nodes is chosen as the correct one and the other versions are discarded. Therefore, an attacker would need to control the majority of the nodes in the network in order to include counterfeit data in the blockchain.

Based on the use of the blockchain, we can distinguish between three `levels'~\cite{Swan2015BlockchainEconomy}:
\begin{itemize}[noitemsep, nolistsep]
    \item \textit{Level 1} is blockchain used with currency only. The data here are transactions of that currency. Level 1 of blockchain would be for example Bitcoin.
    \item \textit{Level 2} includes smart contracts and more advanced transactions and agreements than Level 1. However, Level 2 of blockchain is still tied to financial applications in some way. Example of Level 2 blockchain is Ethereum platform.
    \item \textit{Level 3} of blockchain includes usage outside of financial applications, in sectors such as government or health-care~\cite{Swan2015BlockchainEconomy}.
\end{itemize}

% Motivate problems with cryptocurrency? Double spending problem and byzantines general problem? 
% Central bank system - problems?
% Decentralised system - solutions
% mentioned briefly - drawbacks
% narrower definition of blockchain - list of transactions for a cryptocurrency
% broader definition - number of other applications, Blockchain 2.0 and Blockchain 3.0 as described by Swan
% 
\subsection{Bitcoin}
% 
Created in 2009, Bitcoin is the first ever digital currency, that operates without a central authority in a completely decentralised manner. It is a cryptocurrency with largest market capitalisation\footnotemark and probably the most famous cryptocurrency worldwide.
% 
\footnotetext{Over USD 115 billion as of 01-04-2018. \url{https://coinmarketcap.com/}, accessed 01-04-2018
}
% 
Bitcoin was proposed by person or group under a pseudonym Satoshi Nakamoto, whose identity is not known to date \cite{Feins2017SatoshiBitcoin}. The initial proposal consisted of a white-paper describing the system \cite{NakamotoBitcoin:System} and the first implementation, written in  C++ \footnotemark.
% 
\footnotetext{Original repository on SourceForge has been moved to Github. \url{https://github.com/bitcoin/bitcoin/tree/4405b78d6059e536c36974088a8ed4d9f0f29898}, accessed 01-04-2018}

In the white-paper, Nakamoto proposes a decentralised currency, based on a proof-of-work blockchain. To include a new block in the blockchain, certain amount of work needs to be carried out by the node. This is a protection against the attempts to include counterfeit data in the blockchain. As discussed earlier, to falsify a past block, an attacker would need to recalculate all the subsequent blocks. Furthermore, they would need to provide proof-of-work for all the subsequent blocks. Since the proof-of-work is computationally intensive, it would be practically impossible for the attacker to outpace the honest nodes.

In case of Bitcoin, the proof-of-work consist of finding such a hash value, that is below a given constant (in other words, such a hash value, that has specified amount of leading zeros). In Bitcoin, this hash value is computed over the hash value of previous block, timestamp\footnotemark, root hash of transactions and random number, called \textit{nonce}. While trying to fulfil the proof-of-work condition, the nodes randomly generate a new nonce and compute a new hash. If this fulfils the leading-zeros requirement, a valid block is produced and can be broadcasted to other nodes. Figure \ref{fig:blocks-bitcoin} shows the composition of the block in detail.
% 
\footnotetext{The timestamp is a local UNIX time of the node. However, this timestamp does not to be very accurate (approximate allowed accuracy is $\pm$ 1 hour). It can happen, that the timestamps in blocks are not in order. The goal of the timestamp is to increase the difficulty of forging the blocks. \url{https://en.bitcoin.it/wiki/Block_timestamp}, accessed 01-04-2018}
% 
\begin{figure}[ht]
    \centering
    \includegraphics[width=.95\textwidth]{blocks-bitcoin}
    \caption{}
    \label{fig:blocks-bitcoin}
\end{figure}

\subsubsection{Alt-coins}

% Maybe mention slightly
% 
\subsection{Ethereum}

Ethereum is a decentralised computing platform, that runs small programs called \textit{smart contracts}. It builds on similar concepts as Bitcoin, such as blockchain and consensus based on proof-of-work. Smart contracts are written in Ethereum Virtual Machine bytecode and executed by the \acrfull{evm}. Ethereum was proposed in 2013 by cryptocurrency researcher Vitalik Buterin and was launched in an \acrlong{ico}\footnotemark  in 2014. 
% 
\footnotetext{In an \acrfull{ico} a small percentage of cryptocurrency is sold to a small percentage of backers in a capital-raising process. \url{https://www.investopedia.com/terms/i/initial-coin-offering-ico.asp}, accessed 10-04-2018}
% 
It is currently developed by a Swiss non-profit organisation \textit{Ethereum Foundation}.\footnotemark
% 
\footnotetext{\url{https://www.ethereum.org/foundation}, accessed 10-04-2018}

\subsubsection{Principle of operation}
While in Bitcoin the blockchain only keeps track of unspent transaction outputs, Ethereum blockchain is more complex. It does involve transactions of own currency, which is called \textit{Ether}, but it goes further and introduces other functionality. It is better described as ``\textit{a state machine, where the distributed nodes maintain a shared view of a global state}''~\cite{Tikhomirov2018Ethereum:Perspectives}. Ethereum users issue transactions, which are distributed among the nodes and which change the state of said state machine. Users can make transactions for one of the following reasons~\cite{Atzei2017ASoK}:
\begin{itemize}[noitemsep]
    \item Create a smart contract.
    \item Invoke a function of a smart contract.
    \item Transfer Ether to a smart contract or to another user.
\end{itemize}

Since \acrshort{evm} bytecode is a Turing complete language~\cite{Tikhomirov2018Ethereum:Perspectives, Atzei2017ASoK, Dannen2017IntroducingSolidity}, smart contracts could possibly perform wide variety of computations. However, this has limited use in practice. Every time a transaction invokes a smart contract function, all nodes that observed this transaction, will run the function of the smart contract at once. This redundancy is a desired feature of a decentralised computing platform, such as Ethereum~\cite{EthereumCommunityEthereumDocumentation}, but could lead to a lot of wasted resources, if large quantities of useless computations are to be performed on many nodes at the same time.

To prevent \acrfull{dos} attacks caused by smart contracts containing endless loops or very intense computations, there is a price associated with every atomic instruction, which in Ethereum terminology is called \textit{gas}. When a transaction is made, it must specify the maximum amount of gas that can be used to compute the code of a smart contract. If the code of the smart contract includes many computations, the contract will use more gas. If it does not finish computing before it reaches the maximum amount of gas, any changes made by the computations will be reverted, but the gas will not be returned. To prevent this from happening, users might set the maximum gas limit high enough, so that the smart contract will not run out of gas during its operation. However, the code of the smart contract is computed on all the nodes participating in the network. To avoid all these nodes carrying out huge amount of (potentially useless) computation, the gas is not free and needs to be purchased\footnotemark. 
% 
\footnotetext{The term \textit{gas} was not chosen coincidentally. It is supposed to resemble the gas we put in our car. The metaphor is as follows: \textit{Our smart contract is the car and the computations performed by the smart contract is the distance driven. If we want to drive further, the car will use more gas. We pay some price per litre of gas, when we stop at the gas station. If we want to drive further, we need to purchase more gas.}}

The gas is purchased at the time when a transaction is made and its price is determined by the free market~\cite{EthereumCommunityEthereumDocumentation}. If an attacker was to invoke an endless loop in a smart contract, the loop would only execute while having enough gas to do so. To spam the network with computationally intense transactions, an attacker would need to purchase high amount of gas and the attack would become too expensive~\cite[p.5]{Atzei2017ASoK}.

\subsubsection{Proof of work}
Similarly as Bitcoin, Ethereum uses proof-of-work to reach consensus. Nodes verify incoming transactions and include them together in a block. This block is then hashed using Ethash, a memory-intensive hash function~\cite[p. 211]{Tikhomirov2018Ethereum:Perspectives}. The hash of the block needs to be below target value to be considered valid. Similarly as in Bitcoin, this process is called mining.  A memory intensive hash function was chosen to target \acrshort{gpu} as the main mining hardware, in order to avoid high barriers for entry. Since \acrshort{gpu} market segment is mainly targeted at other areas, such as gaming or high-performance computing, powerful hardware is already available at the market\footnotemark. The target value for the block hash is adjusted regularly to keep average block production speed constant~\cite{Tikhomirov2018Ethereum:Perspectives}.
% 
\footnotetext{In Bitcoin and other currencies, specialised mining hardware, also referred to as \acrfull{asic} is mainly used for Bitcoin mining. The cost of such hardware and economies of scale make it difficult for small miners to enter the market, therefore weakening one of the main advantages of Bitcoin -- decentralisation~\cite{Tikhomirov2018Ethereum:Perspectives}.}

\subsubsection{Transactions and accounts}
There are two types of accounts defined in Ethereum:
\begin{itemize}[noitemsep]
    \item \textit{\acrfull{eoa}}. \acrshort{eoa}s are accounts controlled by a public-private key-pair that have a non-zero Ether balance. \acrshort{eoa}s can send transactions.~\cite{EthereumCommunityEthereumDocumentation}
    \item \textit{Contract account}. Contract account is an account that has code associated with it. It can run its code and maintain its internal state. It can communicate with other contracts by sending objects called \textit{messages}. Execution of contract's code is triggered by an incoming transaction or a message~\cite{EthereumCommunityEthereumDocumentation}.
\end{itemize}

\subsubsection{Smart contracts}
\acrshort{evm} bytecode is a low-level language, which natively supports cryptographic primitives~\cite{Tikhomirov2018Ethereum:Perspectives}. Smart contracts are usually written in high-level language and are compiled into \acrshort{evm} bytecode before deployment. Most popular language for writing smart contracts is Solidity, which offers JavaScript-like syntax. Solidity compilers exist for various platforms, including a web-based \acrshort{ide}.

Smart contracts usually cannot perform very extensive calculations, as the \acrshort{evm} is not powerful enough and there is a associated with all computations performed by the smart contract. Because of this, every smart contract is either used for some sort of financial transaction or at least requires some funds to be transferred to it, so its code can execute. A smart contract can be a building block in the overlaying concept of \textit{decentralised application}. The scope of decentralised applications is bigger than that of smart contracts. Some decentralised application may operate their own blockchain, while others may use existing blockchains (such as Ethereum)~\cite{AlyssaHertigWhatCoinDesk}. According to a curated repository over the Ethereum network, there are many decentralised applications ``\textit{many projects covering different fields such as health, Ponzi schemes, games, virtual reality, artificial intelligence, education, registries, job markets, [...] and many more}''\footnotemark. 

\footnotetext{\url{https://www.stateofthedapps.com/about}, accessed 18-05-2018}

\subsection{Nomenclature}
Despite the existing definitions for words \textit{currency}, \textit{fiat currency} and \textit{cryptocurrency}, there are several naming convention in use today. This paragraph describes, which terms will be used further in this report.

The value of many cryptocurrencies is not backed by any commodity and it depends entirely on the supply/demand relationship. Therefore, by the definition, these cryptocurrencies are also fiat currencies. However, in the literature, the term \emph{fiat currency} is often used when referring to traditional, government issued currencies only. For the rest of this paper we will therefore keep this naming. Regular currencies, such as dollars, euros or crowns will be referred to as fiat currencies.

Virtual, digital money such as Bitcoins, Ether, Litecoin and similar are in minority of sources named `coin currencies'. However, in this paper they will be referred to as \textit{cryptocurrencies} instead, as this is the most common term.