\subsection{Security-by-design}

Generally speaking, it is usually cheaper to make a change to a system during development, rather than after it has been deployed~\cite[p. 9]{Sommerville2011SoftwareEngineering}. This also applies to security. Cases where system vulnerabilities lead to loss of money or data happen on regular basis -- the funds stolen from the cryptocurrency exchange Mt. Gox~\cite{Popper2014ApparentTimes} or a cyber-attack leading to loss of USD 300 million at the shipping company Maersk~\cite{JordanNovet2017MaerskMillion} are just two of the many examples.

If a system operates with money, the need of proper security measures is even more obvious. In this project, we base our motivation on a fact that users can usually not verify the software used by a cryptocurrency exchange company. Some companies may conceal how their systems operate to protect their software from competitors, but others may simply rely on security-by-obscurity (security-by-obscurity is an approach that bases the security of a system on the secrecy of its implementation and is not recommended~\cite{Scarfone2008GuideTechnology}). Either of those two options is the reason, the system is still closed and users have no way to verify, whether the system is secure or not. Rather than that, we adopt the opposite approach. Openness of the system, which can be reviewed and security flaws can be identified by anyone.

Another aspect of security-by-design is expecting malicious practices while designing the system. Attacks on the system are taken as granted, rather than possible. This provides aid when designing the system architecture and helps maintain the \textit{principle of least privilege}. This principle states, that \textit{``a user should be given no more access to resources than it is required to complete the task at hand''}~\cite{XiaopuSpecifyingControl}. 

We follow these principles, when we design the system architecture and expect that an adversary will attempt to steal funds of the users, during the system operation. This helps us with the decision making process in the \textit{Architecture} part.