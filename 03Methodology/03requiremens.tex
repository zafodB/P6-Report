\subsection{Requirement engineering}

Requirement engineering is the process of defining and documenting requirements of a system~\cite{Sommerville2011SoftwareEngineering}. Requirements can be written down with various level of detail, ranging from high-level, general descriptions of the system (e.g. \textit{A system should be able to generate a report}) to low-level detailed rules about the properties of a system (e.g. \textit{A system should generate a report in PDF format within 2 minutes after a user has pressed a button}). Requirements can be also classified into functional and non-functional requirements. In short, the functional requirements describe what the system should do, while the non-functional requirements describe, how the system should perform~\cite{Sommerville2011SoftwareEngineering}. As Sommerville notes, it may be difficult to separate functional and non-functional requirements, as these are necessarily related and may overlap. Non-functional requirements can be used to specify the measurable performance criteria of a system in terms of speed, reliability, size and so on. However, they can also be used to specify properties that can not be directly measured, but are not describing system functionality either (e.g. legal or ethical requirements)~\cite{Sommerville2011SoftwareEngineering}. 

As noted in the previous section, we use requirements in two places of this report. Firstly, we use high-level requirements to describe the properties of the proposed system in general. Later, we use requirements again to describe properties of individual system components. We use both functional and non-functional requirements.Our functional requirements describe, how the system is to be used and what functionality needs to be included. The non-functional requirements mostly resolve around security of the system, rather then performance or efficiency.

It is also important to note, that in this project, all the requirements come from the developers and none from the users. This would not be the case in a regular software development project as the end users are traditionally involved in the requirement elicitation~\cite{Kujala2005TheSuccess}.