\subsection{Prototype requirements}

In this section, we will describe the desired behaviour of the prototype using functional and non-functional requirements. In the first part, functional requirements are used to describe the intended functionality of the system as a whole, as seen from user's perspective. In the second part, we will capture some important characteristics that the system should have, using non-functional requirements.

\subsubsection{Functional requirements}
Usually, functional requirements would come from the users of the system, product owner and other stakeholders. In this project, the authors took the role of the product owner and envisioned the first set of requirements. This first set made up the prototype requirements. In further releases, users and other stakeholders would likely need to be considered.

% we list the requirements for the prototype, so that the idea can be demonstrated clearly. Requirements necessary for demonstrating the idea were prioritised, the others were not worked upon. We ranked the requirements by their importance and then sorted them in a MoSCoW table. We used this table to prioritise the areas for implementation. In a market-ready product, many requirements would most likely be different.

The idea of the system is to allow users trade Ether for another cryptocurrency, which is Bitcoin in this case. Both the user that sells Ether and buys Bitcoin (in previous chapter also referred to as \textit{Alice}) and the user that buys Ether and sells Bitcoin (previously referred to as \textit{Bob}) interact with the same application. For clarity, for the rest of this chapter, we will refer to the Ether selling user as \textit{primary user} and their trading partner (Ether buying user) as \textit{secondary user}.

The requirements for the prototype were selected in such way, so that the overall idea of the system can be demonstrated. The selection is demonstrated in Table \ref{tab:pt-func-reqs}. 

\begin{table}[ht]
    \centering
    % \begin{framed}
    \begin{tabularx}{\textwidth}{|l X|}
    \hline
    \textbf{Requirement ID}& \textbf{Description}\\
    &\\
    PT-FR-1 & The primary user and the secondary user should be able to agree on the terms of the transaction.\\
    PT-FR-2 & The primary user must be able to deploy a smart contract to the Ethereum network.\\
    PT-FR-3 & The secondary user should be able to verify contract deployment.\\
    PT-FR-4 & The secondary user should be able to transfer Bitcoin.\\
    PT-FR-5 & If the secondary user does not transfer Bitcoin, the primary user must receive their Ether back.\\
    \hline
    \end{tabularx}
    % \end{framed}
    \caption{Initial set of functional requirements on the prototype.}
    \label{tab:pt-func-reqs}
    \end{table}

% 
\subsubsection{Non-functional requirements}
The overall goal is to develop a secure solution, that does not allow a malicious actor to gain control over other users' funds. To achieve this, the architecture envisioned in Figure \ref{fig:arch-ver2-tech} (p. \pageref{fig:arch-ver2-tech} should be used to carry out the transaction. Furthermore,centralisation the system should be avoided, so that the users can continue using the system even if any particular provider goes offline or is attacked. Table \ref{tab:pt-func-reqs} lists these non-functional requirements.

\begin{table}[ht]
    \centering
    \begin{tabularx}{\textwidth}{|l X|}
    \hline
    \textbf{Requirement ID}&\textbf{Description}\\
    &\\
    PT-NFR-1&The system must implement architecture proposed in Figure \ref{fig:arch-ver2-tech} (p. \pageref{fig:arch-ver2-tech}.\\
    PT-NFR-2&There should not be single point of failure in the system.\\
    PT-NFR-3&No other party should have control over the funds of primary user and secondary user.\\
    PT-NFR-4&Primary user and secondary user must be ably to verify the status of their transaction.\\
    \hline
    \end{tabularx}
    
    \caption{Non-functional requirements drawn from the initial idea.}
    \label{tab:pt-nonfunc-reqs}
\end{table}

The design of the final architecture, described in the following section is based on these functional and non-functional requirements. The design of the individual system components also takes these requirements into consideration and uses them to derive individual requirements for each component.

%%%%%%%%%%%%%%%%%%%%%%%%%%%%%%%%%%%%%%%%%%%%%%%%%%%%%%%%%%%%%%%%%%%%%%%

% Using FR to describe how the system will operate
 
% begin by FR of the prototype: what functions are included? what parts of the idea are "business critical" - contract deployment
% what it must do
% what it might do
% what it won't do (but should be in the market ready version)

% then NFR of the prototype: how does the system behave
% reqs - no single Point of failure
% reqs - no "last minute pull outs" - once transaction is made, it is final
% reqs - security - 3rd party cannot steal funds if part of the system fails
% reqs - transparency - users need to be able to verify the process


% % Then:
% % each component:
% % Functional reqs: what it does

% % Non-functional reqs - how does it help to achieve the overall NFRs

% //Overall requirements

% //HERE: MoSCoW with requirements (probably?)
% //Maybe: detailed requirements for each component separately