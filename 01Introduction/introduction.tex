\section{Introduction}
% 
Exchanging cryptocurrencies is becoming a regular task for an increasing number of people. There are countless different platforms that offer this service, ranging from simple wallets that offer currency exchange as additional service, to massive trading platforms with daily market cap of millions of dollars. All of these exchanges, however share one common feature - when making a transaction, we need to trust them with our funds. Let us consider an imaginary cryptocurrency exchange website \textit{Exchange.net}. This website offers exchange from Bitcoin to Litecoin and vice versa. Whenever we wish to use Exchange.net, we need to transfer our funds to the company's account, exchange the funds on the website and then withdraw our exchanged funds again. Until we make the withdrawal, Exchange.net has full control over our money. If the company suddenly goes out of business, or gets attacked, we may never get our funds back! There have been various cases, where deposited funds were lost or stolen, with the most famous case of the Japanese-based Bitcoin exchange Mt. Gox, where USD 380 million worth of Bitcoins were stolen in 2014~\cite{Popper2014ApparentTimes}.

There are numerous reasons to buy cryptocurrency, ranging from a value holding purposes to short-term trading. However, the trust in the cryptocurrency is somewhat unavoidable. Regardless of our motivation, whenever we acquire a (crypto-) currency, we believe that it will retain value at least until we exchange it for other services or goods. Researchers and developers have been trying to create a cryptocurrency for a long time. As early as 1983 anonymous digital money was proposed by an American cryptographer David Chaum~\cite{Chaum1983BlindPayments}. 

Today's cryptocurrencies, such as Bitcoin and Litecoin are thus only the continuation of a long lasting effort to develop a currency with sufficient guarantees against misuse, such as double-spending or theft. The wide-spread adoption of these currencies indicates, that the guarantees against misuse are indeed sufficient -- in other words, that the users trust the cryptocurrency. If this was not the case, it probably would not be used. On the other hand, the trust in the exchange platform is more questionable. Some advise not to store any funds in the exchange for longer periods of time and to make deposits to a secured wallet right away - because it is believed the funds are more vulnerable in an exchange~\cite{McIntosh2018HowScams}.

In this project we address this problem and propose a system, where the trust in the currency is used for making the exchange process of exchange safer. The Ethereum system, which is the second most popular cryptocurrency today\footnotemark could be a good place to investigate. Ethereum gives the possibility of running \emph{smart contracts} -- small programs that are run by many computers around the world at the same time. This provides assurance, that no-one will tamper with the code and that no-one can manipulate with the outcome of a smart contract. 
% 
\footnotetext{By market cap. \url{https://coinmarketcap.com/all/views/all/}, accessed 21-05-2018}
% 
In this project, we will investigate, if smart contracts could be used to facilitate cryptocurrency exchange and how such system could operate. The rest of the report is structured as follows: we first define the problem in the \textit{\ref{sec:problem-formulation} Problem Formulation} chapter. Afterwards, we describe the methods and processes, used in this project in the \textit{\ref{sec:methodology} Methodology} chapter and introduce the basic concepts of cryptocurrency exchange in \textit{\ref{sec:background} Background}. Later, we investigate the existing systems that are relevant to our proposal in the \textit{\ref{sec:SOTA} State of the Art} chapter, before we describe considerations for the system architecture in the \textit{\ref{sec:system-design} System Design}. Finally, we elaborate on the requirements and system implementation details, together with a short evaluation and a model scenario in the \textit{\ref{sec:implementation} Implementation and Evaluation} chapter. The outcomes of this project are debated in \textit{\ref{sec:discussion} Discussion}, before the report is concluded in the \textit{\ref{sec:conclusion} Conclusion}. This report has 4 appendices.