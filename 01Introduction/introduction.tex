\section{Introduction}
% 
Exchanging cryptocurrencies is becoming a regular task for an increasing number of people. There are countless different platforms that offer this service, ranging from simple wallets that offer currency exchange as additional service, to massive trading platforms with daily market cap ranging to millions of dollars. All of these exchanges, however share one common feature - when making a transaction, we need to trust them with our funds. There have been various cases, where deposited funds were lost or stolen, with the most famous case of the Japanese-based Bitcoin exchange Mt. Gox, where USD 380 million worth of Bitcoins were stolen in 2014 \cite{Popper2014ApparentTimes}.

The trust in the cryptocurrency is somewhat unavoidable - if we did not trust the currency, we would probably not buy it in the first place. On the other hand, the trust in the exchange platform is more questionable. Some advise not to store any funds in the exchange for longer periods of time and to make deposits to a secured wallet right away - because they believe the funds are more vulnerable in an exchange \cite{McIntosh2018HowScams}.

What if we tried to leverage the trust in the currency into making the process of exchange safer? The Ethereum platform could be a good place to investigate. Ethereum gives the possibility of running \emph{smart contracts} - small programs that are run by many computers around the world at the same time. This provides assurance, that no-one will tamper with the code and that no-one can influence the outcome of a smart contract. 
% 
% DUNNO WHAT TO WRITE. wILL REVISIT THIS SECTION
% 

In this project, we will investigate, if smart contracts could be used to facilitate cryptocurrency exchange and how such system could operate.