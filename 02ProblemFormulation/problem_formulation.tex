\section{Problem Formulation}\label{sec:problem-formulation}
% 
The introduction presents a problem with the current currency exchange scheme. Our preliminary research question, trying to solve the problem was ``\textit{How can trust in a third-party exchange provider be replaced with trust in the open cryptocurrency system?}''. However, we quickly realised that this was too broad and could possibly include countless different solutions. Our interest brought us to Ethereum -- as mentioned previously, Ethereum could possibly enable such a platform. This brought us to our second version: ``\textit{How can smart contracts be used to solve the trust problem?}'' Upon preliminary research on the idea, it was clear to us, that a smart contract alone could not solve the problem. We therefore reformulated our problem formulation and reached the following question:

\begin{itemize}
    \item \textit{How to implement an open, decentralised cryptocurrency exchange which operates on a decentralised ledger system?}
\end{itemize}

\noindent The motivation for this project stems from the realistic situation, that two people who wish to trade cryptocurrencies, do not currently have the means on how to engage in a trade, without trusting a third party. While it is obvious that not all cryptocurrency exchanges are inherently malicious, they are an external factor that needs to be involved in a transaction. Naturally, users may just trade directly with each other, but this carries a risk, that one of them does not keep their part of the agreement and does not transfer their funds.

We propose a solution, that acts as an intermediary between two trading user, but is embedded in a distributed way in the cryptocurrency system. The logic of this solution needs to be open to anyone, so that whoever wishes to trade, can inspect how the system operates and can make an informed decision, whether such system is to be trusted or not. Our goal is to implement such system in a decentralised manner, so that there is not a single entity that could prevent or manipulate the transaction process.