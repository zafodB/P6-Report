\section{Discussion}\label{sec:discussion}

In this project, we propose a process, which if followed correctly, can establish a secure trade between two parties. We successfully implemented a prototype and demonstrated the process, using this prototype. It is common understanding that the prototype only demonstrates a part of the functionality and does not perform as well as the market-ready product. However, even if we fixed all the weaknesses described in the previous section and launched the product to the market -- would it be used?

The answer is not a clear ``Yes'' as we might wish. There are couple of challenges that need to be considered and that might deter possible users from actually using the system. The cost of the transaction, contract verification and contract deployment approach are some of those challenges and are described in the following paragraphs.

\subsection{Cost of a transaction}
As mentioned earlier, there is a cost associated with every smart contract. The deployment of the smart contract used in the prototype uses approximately 2 million gas (as a comparison, simple transfer of Ether uses exactly 21,000 gas). The price we pay for this amount of gas depends on the market conditions and on how fast do we want the transaction to be processed. Currently, the gas price averages around 14 Gwei\footnote{1 ETH = 1,000,000,000 Gwei} per unit of gas for an average transaction\footnotemark. The cost of a contract deployment is therefore 2 million * 14 Gwei = 28,000,000 Gwei = 0.028 ETH. The approximate price of transaction is around EUR~15, which is quite high, when compared with the fees usually charged by cryptocurrency exchanges -- usually below 1\% of the transaction amount. The transaction cost of EUR~15 is probably reasonable on large amounts, but on regular every-day transactions it may be too high.
% 
\footnotetext{Source: \url{https://ethgasstation.info/}, accessed 22-05-2018}

Limiting the amount of work preformed on the \acrshort{evm} by the smart contract can reduce the cost of the contract deployment and therefore the costs of the transaction. It is also important to note that the smart contract is extending a class provided by Oraclize. It is possible that using another provider as the oracle could reduce the price for the smart contract deployment.

\subsection{Contract verification}
The transaction process, outlined in this report relies on the fact that the secondary user needs to verify the smart contract, after it has been deployed. While the smart contract can be found on the blockchain, using the blockchain explorer, it is only the compiled bytecode that is being stored. If the user does not have reference bytecode or Solidity source code to use for comparison, they would have hard time determining, whether the deployed smart contract is actually correct. This could be solved by the developers making the smart contract code publicly available, but even though, it still is a significant burden for the end user. However, if they do not verify that the deployed smart contract is correct, they could possibly lose their funds.

\subsection{One contract per transaction}
The current prototype always deploys a new contract, whenever a transaction is made. This contract is then executed and discarded afterwards. Another approach would be to only deploy a single contract for the whole system. New transactions would simply call methods defined by this contract and the contract would be handling multiple transactions simultaneously. Naturally, the smart contract would need to be more complex, but the costs for its operation would be shared among all of its users. This approach is also used on the market -- the trading platform IDEX, which we investigated in the State of the art uses a single contract for its operation.
