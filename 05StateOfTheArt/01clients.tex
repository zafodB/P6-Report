\subsection{Ethereum clients //work title}
% 
To communicate with other peers in the Ethereum network, an Ethereum client is needed. Ethereum client implements the JSON-RPC protocol to talk to its peers

There are three reference Ethereum protocol implementations available:
\begin{enumerate*}[label=(\roman*)]
    \item \textit{Go Ethereum}\footnote{\url{https://geth.ethereum.org/}, accessed 06-05-2018}, written in Go
    \item \textit{Pyethereum}\footnote{\url{https://github.com/ethereum/pyethereum}, accessed 06-05-2018}, written in Python and
    \item \textit{Cppethereum}\footnote{\url{https://github.com/ethereum/cpp-ethereum}, accessed 06-05-2018}, written in C++.
\end{enumerate*}
 and number of competing implementations.
 
\begin{figure}[ht]
    \centering
    \begin{tabular}{|l|c|c|c|l|}
        \hline
        \textbf{Name} & \textbf{\# Watched} & \textbf{\# Starred} & \textbf{\# Forked} & \textbf{Developer}\\
        \hline
        \hline
        go-ethereum & 1779 & 17314 & 5516 & Ethereum Foundation\\
        \hline
        pyethereum & 292 & 2330 & 696 & Ethereum Foundation\\
        \hline
        cpp-ethereum & 529 & 3030 & 1901 & Ethereum Foundation\\
        \hline
        parity & 327 & 4164 & 826 & Parity Technologies\\
        \hline
        ethereumj & 220 & 1505 & 775 & Roman Mandeleil\\ 
        \hline
        ruby-ethereum & 49 & 21 & 37 & Jan Xie\\
        \hline
    \end{tabular}
    \caption{Caption}
    \label{fig:my_label}
\end{figure}
 
The most notable include \textit{Parity}, written in Rust\footnote{\url{https://parity.io/}, accessed 06-05-2018} and ... //Write more
 
 Since all of the mentioned clients are open-source and available on GitHub, we can use these stats to approximately compare their popularity.

As of the time of writing, there are no native Ethereum clients for Android.

//Write more