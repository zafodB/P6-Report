\subsection{Cryptocurrency exchanges}

Unless a person is running a node that participates in the mining process, the only other option when they want to get some cryptocurrency, is to buy it from someone else. There is a multitude of options to purchase cryptocurrency, including cryptocurrency ATMs, companies that allow purchase of cryptocurrency with a PayPal account or a credit card to dedicated forums where users trade with each other to specialised websites that list local people, who are willing to trade cryptocurrency for cash. However, the most popular option is still to use a cryptocurrency exchange. 

The principle of a cryptocurrency exchange is, that a person starts by registering on the website of the cryptocurrency exchange. Then they proceed by depositing some amount in fiat currency to the exchange, which can be done using a credit card, PayPal, bank transfer or other means. The amount transferred to the exchange in fiat currency is then credited to this person's account. The person can use their account to exchange this credit for other crypto- or fiat currencies. At the end, the person can withdraw their funds from the exchange to their own private wallet (in case of cryptocurrencies) or bank account (in case of fiat currencies).

While the currency exchange process is usually the same, the cryptocurrency exchanges differ in fees, deposit and withdrawal speed or features. Typically, the fees range from 0.05\% to 1.5\% of the transaction volume, per transaction, with additional possible fees for deposit and withdrawal\footnotemark. 
% 
\footnotetext{\url{https://crowdsourcingweek.com/blog/bitcoin-exchange-comparison/}, accessed 22-05-2018}
% 
Some exchanges require identity verification, before they allow users to trade, to prevent frauds and money laundering.\footnotemark
% 
\footnotetext{\url{https://www.bitstamp.net/faq/\#how-to-verify-your-personal-account}, accessed 22-05-2018}

Some of the most popular cryptocurrency exchanges are: \textit{Coinbase}, \textit{Bitstamp}, \textit{CEX.IO}, \textit{eToro} or \textit{Binance}\footnotemark.
% 
\footnotetext{\url{https://www.bestbitcoinexchange.io/}, accessed 22-05-2018}